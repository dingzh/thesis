% !TeX root = ../thesis.tex

%# -*- coding: utf-8-unix -*-
%%==================================================
%% abstract.tex for SJTU Master Thesis
%%==================================================

\begin{abstract}
随着多种公有云和私有云架构的发展和成熟,越来越多的网络应用和计算任务被迁移至数据中心执行。数据中心建造和运行成本高昂,其中50\%以上的花费用于购买服务器硬件。然而,其服务器的平均资源利用率却通常不超过20\%。造成资源利用率低下的主要原因是大量延迟敏感型任务的存在。处理器中的计算核心共享多种计算资源,因此,并行执行的任务间存在资源竞争并将相互干扰。延迟敏感型任务受到严格的服务质量要求限制,对共享资源上的干扰十分敏感;为保证其服务质量,通常禁止其他任务与之混合执行。基于软件和硬件的资源隔离技术可以有效限制共享资源上的相互干扰,使得延迟敏感型任务与其他任务的混合执行成为可能,其有效应用的一大挑战在于多种共享资源间产生的相互干扰。本文中,我们尝试在延迟敏感型任务与其他任务混合执行的情境下,应用强化学习方法来协调多种共享资源的隔离,在保证延迟敏感型任务的服务质量的前提下,提高服务器整体的资源利用率。


\keywords{\large 数据中心 \quad 资源管理 \quad 强化学习}
\end{abstract}

\begin{englishabstract}

As public and private cloud frameworks develop and mature, an increasing number of workloads are migrating to large-scale datacenters. Typical datacenters often cost tens of million dollars, and servers are the largest fraction. However, the server utilization in most datacenters is low, often not exceeding 20\%. A primary reason for the low utilization is the popularity of latency-critical services. Each server houses several cores, and a number of resources are shared among these cores. As tasks on the same machine actively compete for shared resources, performance interferences arise. Due to strict quality-of-service(QoS) requirements, the latency-critical services are extremely sensitive to interferences. To avoid QoS violation, co-location is often disallowed for latency-critical applications in practice. Recently introduced resource isolation mechanisms allow us to mitigate these interferences.  Aiming to increase server utilization with co-location while avoiding QoS violation, we propose to build a reinforcement learning agent to automatically coordinate multiple isolation mechanisms in modern servers.

\englishkeywords{\large Data Center, Resource Management, Reenforcement Learning}
\end{englishabstract}

